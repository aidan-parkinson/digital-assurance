\documentclass[11pt, oneside]{article}   	% use "amsart" instead of "article" for AMSLaTeX format
\usepackage{geometry}                		% See geometry.pdf to learn the layout options. There are lots.
\geometry{a4paper}                   		% ... or a4paper or a5paper or ... 
%\geometry{landscape}                		% Activate for rotated page geometry
%\usepackage[parfill]{parskip}    		% Activate to begin paragraphs with an empty line rather than an indent
\usepackage{graphicx}				% Use pdf, png, jpg, or eps§ with pdflatex; use eps in DVI mode
								% TeX will automatically convert eps --> pdf in pdflatex		
\usepackage{amssymb}
\usepackage{amsmath}
\usepackage{authblk}
\usepackage{graphicx}

\usepackage{url}

%SetFonts

%SetFonts
\title{A Global Digital Assurance Framework for Integrating Essential Ecosystem Features within a Stable, Diverse and Extensive Alarm Management Platform}
\author{Dr Aidan Thomas Parkinson\\ Realfeed Ltd., Sinsuran, West End, Wedmore, Somerset, BS28 4BW.\\ aidan.parkinson@gmail.com}
\date{\today}							% Activate to display a given date or no date

\begin{document}
\maketitle

\section*{Abstract}

\pagebreak

\section{Introduction}
Digital platform governance involves thoughtful considerations towards the standardisation of interfaces to enable a diverse market of complementors to find opportunity in developing services for consumers.
There has been much discussion about the creation of platforms that seek network effects to deliver a popular market share of consumers with compelling development opportunities for complimentors to offer proprietary tools to enhance personal utility.
Such an approach has been very successful for exploiting new domains, such as the internet and personal devices.
However, such contributions may be somewhat detremental in addressing societies pressing contractual problems involved in minimum security standards to govern a limited natural ecosystem.
Loss of biodiversity, frameworks to limit global temperature rise, social unrest and military intervention are all relevant demonstrations in this domain and appear to exhbit some common features.

One potential area of conflict in ecosystem governance is the management of competing principles of what is Good.
This involves a definition that not only justifies ecosystem ethical priorities, but also a recognition of what is Sovereign.
Minimal policies could then be identified to contribute towards ecosystem stability and align supply-chains with a natural course of development.

Another important aspect is the identification of Frameworks and Guidance that signal the failure of Features to successfully integrate with an ecosystem.
Failure is an important feedback mechanism for developers to learn and adapt work-in-progress Features and, hence, enforcement need ideally to be rapid, consistent and relevant.
It has been recognised that practical networking of outstations and sensors in the field with public domains has historically been poorly enforced.
Lack of consumer confidence has contributed to sluggish market penetration of internet-of-things technologies to-date.

Further, it appears important to consider the management of Intellectual Property and how governing agents may operate.
Such thought needs to reflect on the ecosystems ethical priorities, competition and operational costs.

Therefore, this article asks the question: "How could one govern an alarm management platform to support a stable and diverse natural ecosystem?"
The discussion draws upon experience of practice and a wide-ranging review of theory.
In conclusion an Ontology that offers a personal perspective on this question is proposed.
It is acknowledged that the problems faced are formidable and no solution can be perfect.
However, it is intended that this article contributes common-sense that may inspire the sustainable development of new technologies going forward.


\section{Ethics}


\section{Failure}


\section{Operations}


\section{Conclusions}

\begin{thebibliography}{99}

\bibitem{a1} Example~A. (2022)
\emph{Something},
Somewhere: Some publisher

\end{thebibliography}

\end{document}  
